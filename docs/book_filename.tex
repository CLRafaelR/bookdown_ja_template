\documentclass[]{book}
\usepackage{lmodern}
\usepackage{amssymb,amsmath}
\usepackage{ifxetex,ifluatex}
\usepackage{fixltx2e} % provides \textsubscript
\ifnum 0\ifxetex 1\fi\ifluatex 1\fi=0 % if pdftex
  \usepackage[T1]{fontenc}
  \usepackage[utf8]{inputenc}
\else % if luatex or xelatex
  \ifxetex
    \usepackage{mathspec}
  \else
    \usepackage{fontspec}
  \fi
  \defaultfontfeatures{Ligatures=TeX,Scale=MatchLowercase}
\fi
% use upquote if available, for straight quotes in verbatim environments
\IfFileExists{upquote.sty}{\usepackage{upquote}}{}
% use microtype if available
\IfFileExists{microtype.sty}{%
\usepackage{microtype}
\UseMicrotypeSet[protrusion]{basicmath} % disable protrusion for tt fonts
}{}
\usepackage[margin=1in]{geometry}
\usepackage{hyperref}
\hypersetup{unicode=true,
            pdftitle={book\_title},
            pdfauthor={author\_name},
            pdfborder={0 0 0},
            breaklinks=true}
\urlstyle{same}  % don't use monospace font for urls
\usepackage{natbib}
\bibliographystyle{apalike}
\usepackage{longtable,booktabs}
\usepackage{graphicx,grffile}
\makeatletter
\def\maxwidth{\ifdim\Gin@nat@width>\linewidth\linewidth\else\Gin@nat@width\fi}
\def\maxheight{\ifdim\Gin@nat@height>\textheight\textheight\else\Gin@nat@height\fi}
\makeatother
% Scale images if necessary, so that they will not overflow the page
% margins by default, and it is still possible to overwrite the defaults
% using explicit options in \includegraphics[width, height, ...]{}
\setkeys{Gin}{width=\maxwidth,height=\maxheight,keepaspectratio}
\IfFileExists{parskip.sty}{%
\usepackage{parskip}
}{% else
\setlength{\parindent}{0pt}
\setlength{\parskip}{6pt plus 2pt minus 1pt}
}
\setlength{\emergencystretch}{3em}  % prevent overfull lines
\providecommand{\tightlist}{%
  \setlength{\itemsep}{0pt}\setlength{\parskip}{0pt}}
\setcounter{secnumdepth}{5}
% Redefines (sub)paragraphs to behave more like sections
\ifx\paragraph\undefined\else
\let\oldparagraph\paragraph
\renewcommand{\paragraph}[1]{\oldparagraph{#1}\mbox{}}
\fi
\ifx\subparagraph\undefined\else
\let\oldsubparagraph\subparagraph
\renewcommand{\subparagraph}[1]{\oldsubparagraph{#1}\mbox{}}
\fi

%%% Use protect on footnotes to avoid problems with footnotes in titles
\let\rmarkdownfootnote\footnote%
\def\footnote{\protect\rmarkdownfootnote}

%%% Change title format to be more compact
\usepackage{titling}

% Create subtitle command for use in maketitle
\newcommand{\subtitle}[1]{
  \posttitle{
    \begin{center}\large#1\end{center}
    }
}

\setlength{\droptitle}{-2em}
  \title{book\_title}
  \pretitle{\vspace{\droptitle}\centering\huge}
  \posttitle{\par}
  \author{author\_name}
  \preauthor{\centering\large\emph}
  \postauthor{\par}
  \predate{\centering\large\emph}
  \postdate{\par}
  \date{2016-12-30}

\usepackage{xltxtra}
\usepackage{zxjatype}
\usepackage[ipa]{zxjafont}

\begin{document}
\maketitle

{
\setcounter{tocdepth}{1}
\tableofcontents
}
\chapter{book\_title}\label{book_title}

このRmdファイルを
\texttt{bookdown::render\_book("index.Rmd")}すると,自動的に製本(?)します。

なお(私の考えうる限りで)最小構成で作ってます。実際に作ろうと思うなら,\href{https://bookdown.org/yihui/bookdown/}{本家ドキュメント}を参照してください。

以下は説明用の文章を貼り付けてます。\textbf{実際には削除してください}。

不明な点があれば,Twitterの\citep{kazutan}(\url{https://twitter.com/kazutan})
もしくはこのリポジトリのissue,あるいはr-wakalangのrmarkdownのチャンネルまでおねがいします。

\section{書籍ファイルの作成方法}

\subsection{必要なパッケージ,環境など}

Knitr, rmarkdown,
bookdownのパッケージがデータのレンダリングに必要です。またpandocの新しいのが必要で,面倒でしたらRStudioの最新版をインストールしてください(内包してます)。
ggplot2逆引き記事内にて使用するパッケージも必要となります。おそらくggplot2パッケージぐらいで大丈夫だと思いますが,面倒でしたらtidyverseパッケージを導入してください。これをインストールするとHadleyverseなパッケージ群が自動的にインストールされます。
もしpdf
bookを作りたいのであれば,マシンにtex環境が必要です。日本語のフォントにIPAフォントを指定していますので,以下からダウンロードしてください。

\url{http://ipafont.ipa.go.jp/}

また,bookdownはutf-8しか受け付けません。そのためwindowsではうまく動かないかもしれません(未検証)。もし何かありましたらissueなりkazutan
までご連絡ください。

私の作業環境(動作確認環境)は,最後にまとめて表示しています。

\subsection{Download}\label{download}

git cloneして持ってくるか,右側のDownload Zipで持ってきてください:

\begin{verbatim}
$ git clone git@github.com:kazutan/bookdown_ja-template.git
\end{verbatim}

\subsection{レンダリング(本のファイル作成)}

\subsubsection{種類}

\begin{itemize}
\item
  gitbook形式: 以下のコードを実行

\begin{verbatim}
bookdown::render_book("index.Rmd", output_format = "bookdown::gitbook")
\end{verbatim}
\item
  epub形式: 以下のコードを実行

\begin{verbatim}
bookdown::render_book("index.Rmd", output_format = "bookdown::epub_book")
\end{verbatim}
\item
  pdf形式: 以下のコードを実行

\begin{verbatim}
bookdown::render_book("index.Rmd", output_format = "bookdown::pdf_book")
\end{verbatim}
\end{itemize}

RStudioを利用しているなら,BuildパネルでBuild
Bookから選択してください。もしBuildタブがRStudioで表示されない場合,一度RStudioを終了させてもう一度開いてください。

\subsection{生成物の場所}

生成物は,\texttt{\_book}ディレクトリに置かれるように設定してます。\texttt{.epub}と\texttt{.pdf}は単独ファイルで,それ以外はgitbook形式のファイルとなります。

\section{session info}\label{session-info}

\begin{verbatim}
R version 3.3.2 (2016-10-31)
Platform: x86_64-apple-darwin13.4.0 (64-bit)
Running under: OS X El Capitan 10.11.6

locale:
[1] ja_JP.UTF-8/ja_JP.UTF-8/ja_JP.UTF-8/C/ja_JP.UTF-8/ja_JP.UTF-8

attached base packages:
[1] stats     graphics  grDevices utils     datasets  methods   base     

loaded via a namespace (and not attached):
 [1] backports_1.0.4 bookdown_0.3    magrittr_1.5    rprojroot_1.1   htmltools_0.3.5
 [6] tools_3.3.2     yaml_2.1.14     Rcpp_0.12.8     stringi_1.1.2   rmarkdown_1.3  
[11] knitr_1.15.1    stringr_1.1.0   digest_0.6.10   evaluate_0.10  
\end{verbatim}

\chapter{章のタイトルをここに入力}

進捗どうですか?

適当に編集してください。

R Markdown and \textbf{knitr} \citep{xie2015}.

\chapter{章のタイトル2}\label{2}

進捗どうですか?

\section{節見出し1}\label{1}

ほげほげ

\section{節見出し2}\label{2}

ふがふが

\bibliography{book.bib}


\end{document}
